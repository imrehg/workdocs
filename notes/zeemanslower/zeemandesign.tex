\documentclass[10pt,a4paper]{article}
\usepackage[latin1]{inputenc}
\usepackage{amsmath}
\usepackage{amsfonts}
\usepackage{amssymb}
\author{Gergely Imreh}
\title{Zeeman slower design}
\begin{document}

\maketitle

\section{Intro}
Writing up the Zeeman slower calculation.

\section{Dimensionless variables}

\begin{center}
\begin{tabular}{|c|c|c|}
\hline 
{\bf Unit} & {\bf Scaling} & {\bf Rationale} \\ 
\hline 
Time & $1 / \Gamma$ & Reciprocal linewidth as relevant interaction parameter\\
\hline 
Distance & $1 / k$ & Reciprocal wavenumber as relevant interaction parameter \\ 
\hline 
Temperature & $T_r = \frac{\hbar^2 k^2}{m k_B}$ & Recoil temperature temperature \\ 
\hline 
Mass & AMU & Atomic mass unit \\ 
\hline 
Magnetic field & from $\mu_B$ & Bohr magneton \\ 
\hline 
\hline 
Current & A = 10^4 \frac{G s^2}{kg} & Here it is derived \\ 
\hline 
\end{tabular} 
\end{center}

Does this really make sense? Is it really necessary??? Actually, all of it can be made to reference a relevant parameter. But this might not be the correct way to do things, should write out the equation of motion and then do the simplification.


\section{Initial parameters}

The magnetic field is inversely proportional to the pitch (proportional to the wire density). To choose initial parameters, we can have:

Pitch is
\begin{equation}
\varphi = \frac{\partial z(p)}{\partial \theta(p)}
\end{equation}
and then
\begin{equation}
B \sim \frac{1}{\varphi} = \frac{\partial \theta(p)}{\partial z(p)} = 
  \frac{\partial \theta(p)}{\partial p}\frac{\partial p}{\partial z(p)}	
\end{equation}
from where 
\begin{equation}
\frac{\partial \theta(p)}{\partial p} \sim B \frac{\partial z(p)}{\partial p}	
\end{equation}
and since $z$ is just a linear function of $p$, with the single relevant parameter $c_7$:
\begin{equation}
\frac{\partial \theta(p)}{\partial p} \sim c_7 B(z(p))
\end{equation}
where the field $B$ can be approximated up to a certain exponential. If $z$ is linear, then $B \sim z^5 \sim p^5$. Choose the length of coil first, that is $L = z(2\pi) - z(0)$,  and position, from there move on to $B$, then $\theta$. The whole thing is shifted by about $R$.


\section{Physics}

Calculating the net gyromagnetic factor:
\begin{equation}
\mu' = \mu_B \left(g_e m_e - g_g m_g\right)
\end{equation}
where $g$ refers to the Lande g-factor of the hyperfine levels ($g_F$), $m$ to the Zeeman sublevel's quantum number, and $e,g$ are excited and grounds states, respectively. Doing the calculation in case of Rb $\mu' = \mu_B$ for all practical purposes for $|m_e - m_g| = 1$.

Most probable velocity:
\begin{equation}
v_p = \sqrt{\frac{2 k_B T}{m}}
\end{equation}

\bibliographystyle{plain}

\end{document}