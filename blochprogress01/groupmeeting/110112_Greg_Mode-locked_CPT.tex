\documentclass{beamer}

\mode<presentation>
\usepackage{beamerthemesplit}
\usepackage{graphics}
\usetheme{Madrid}
%% \usetheme{Boadilla}

\title[Comb CPT]{Coherent population trapping\\with mode-locked laser}
\author{Gergely Imreh}
%% \institute[IAMS]
%% {
%% IAMS, Academia Sinica \\
%% \medskip
%% {\emph{imrehg@gmail.com}}
%% }
\date{January 12, 2011}

\begin{document}

\frame{\titlepage}

\section[Outline]{}
\frame{\tableofcontents}


\section{Others' research}
%% Compare research: Rb, Cs, CW, Mode locked, theory, experiment


\section{The case of caesium}
%% Electronic structure D1, D2 line


\section{Experimental complexity}
%% What effects to take into account




%% \section{Ion trapping}
%% \frame
%% {
%% 	\frametitle{Trapping charged particles}
%% 	\begin{block}<+->{What is the problem?}
%% 		Laplace's equation: $\nabla^2\phi = 0$\\
%% 		We cannot trap with electrostatic fields in all 3 dimensions!
%%  	\end{block}

%% 	\begin{columns}<+->[T]
%%  		\begin{column}{6cm}
%%   			\begin{block}{Solution:}
%% 				\begin{itemize}
%% 					\item Oscillating electric field\\(confining-expelling)
%% 					\item Ion cannot escape due to intertia.
%% 					\item Time averaged field is confining.
%% 				\end{itemize}
%% 			\end{block}
%% 		\end{column}
%% 		\begin{column}{4cm}
%% 			\includegraphics[width=4cm]{pictures/Saddle_point.eps} 
%%  		\end{column}
%%  	\end{columns}
%% }

%% \frame
%% {
%% 	\frametitle{Ion trap design example: the linear Paul trap}
%% 	\begin{columns}[C]
%% 		\begin{column}{5cm}
%% 			\begin{block}{Considerations}
%% 				\begin{itemize}
%% 					\item<+-> \alert<.>{Endcaps}:\\ static confinement along the trap axis.
%% 					\item<+-> \alert<.>{RF electrodes}:\\ oscillating field for radial trapping
%% 					\item<+-> Straightforward assembly,\\large length scale (mm),\\medium-strength confinement\\ (few 100kHz)
%% 				\end{itemize}
%% 			\end{block}
%% 		\end{column}
%% 		\begin{column}{6cm}
%% 			\includegraphics[width=6cm]{pictures/paul_trap.eps} 
%% 		\end{column}
%% 	\end{columns}
%% }

%% \frame
%% {
%% 	\frametitle{Ion trap design example: the linear Paul trap}
%% 	\begin{columns}[C]
%% 		\begin{column}{5cm}
%% 			\begin{block}{Ion trap advantages}
%% 				\begin{itemize}
%% 					\item<+-> \alert<.>{Small coupling to the environment}:\\ free space confinement
%% 					\item<+-> \alert<.>{Large coupling on demand}:\\ laser-ion interaction
%% 					\item<+-> Noise source:\\ fluctuating electric/magnetic fields
%% 				\end{itemize}
%% 			\end{block}
%% 		\end{column}
%% 		\begin{column}{6cm}
%% 			\includegraphics<1>[width=6cm]{pictures/paul_trap.eps} 
%% 			\includegraphics<2->[width=6cm]{pictures/paul_trap_laser.eps} 
%% 		\end{column}
%% 	\end{columns}

%% }



%% \frame
%% {
%% 	\frametitle{The ion's motion in the RF field}
%% 	\begin{columns}[c]
%% 		\begin{column}{5cm}
%% 			\only<1>{
%% 				\begin{flushright}
%% 				Direction of force on a \textbf{+} ion,\\radial cross section of trap $\rightarrow$
%% 				\end{flushright}
%% 			}
%%  		\only<2->{\begin{block}{Motion in the RF field}
%% 			\begin{itemize}
%% 				\item<2-> Ion enters "trapping" region
%% 				\item<3-> Ion moves according\\ the instantenous fields
%% 				\item<4-> Field changes polarity
%% 				\item<5-> Ion needs to change\\ direction
%% 				\item<8-> On average: trapping
%% 				\item<9> Laser cooling:\\ reduce excursion
%% 			\end{itemize}
%% 		\end{block}}
%% 	\end{column}
%% 	\begin{column}{5cm}
%% 		\begin{center}
%% 			\only<1>{\includegraphics[height=4.5cm]{trapping/fieldvectors_1.eps}\\ T = 0}
%% 			\only<2>{\includegraphics[height=4.5cm]{trapping/fieldvectors_2.eps}\\ T = 0}
%% 			\only<3>{\includegraphics[height=4.5cm]{trapping/fieldvectors_3.eps}\\ T = 0}
%% 			\only<4>{\includegraphics[height=4.5cm]{trapping/fieldvectors_4.eps}\\ T =  $\frac{1}{2}$ RF period}
%% 			\only<5>{\includegraphics[height=4.5cm]{trapping/fieldvectors_5.eps}\\ T =  $\frac{1}{2}$ RF period}
%% 			\only<6>{\includegraphics[height=4.5cm]{trapping/fieldvectors_6.eps}\\ T =  1 RF period}
%% 			\only<7>{\includegraphics[height=4.5cm]{trapping/fieldvectors_7.eps}\\ T =  1 RF period}
%% 			\only<8>{\includegraphics[height=4.5cm]{trapping/fieldvectors_8.eps}\\ Long term movement}
%% 			\only<9>{\includegraphics[height=4.5cm]{trapping/fieldvectors_9.eps}\\ With cooling}
%% 		\end{center}
%% 	\end{column}
%% \end{columns}

%% }


%% \frame
%% {
%% 	\frametitle{Trap parameters}
%% 	\begin{columns}[c]
%% 		\begin{column}{6cm}
%% 			\begin{block}{RF Pseudo-potential}
%% 				\begin{itemize}
%% 					\item<1-> RF potential:\\$\phi(\vec{r},t) = V_{\rm rf}\phi_{\rm rf}(\vec{r})\cos(\Omega_{\rm rf} t)$\\
%% 					\item<.->
%% 					\item<2-> Time-average pseudo-potential:\\$U_p(\vec{r}) = \frac{q^2 V_{\rm rf}^2}{4m\Omega_{\rm rf}^2}|\nabla \phi_{\rm rf}(\vec{r})|^2$\\
%% 					\item<.->
%% 					\item<3-> Limits on suitable $\Omega_{\rm rf}$ values
%% 				\end{itemize}
%% 			\end{block}
%% 		\end{column}
%% 		\begin{column}{4cm}
%% 			\begin{center}
%% 				\only<1>{\includegraphics[height=5.5cm]{trapping/pseudopot.eps}}
%% 				\only<2->{\includegraphics[height=5.5cm]{trapping/pseudopot2.eps}}
%% 			\end{center}
%% 		\end{column}
%% 	\end{columns}

%% }

%% \frame
%% {
%% 	\frametitle{Choices of ions}
%% 	\begin{columns}[c]
%% 		\begin{column}{4cm}
%% 			\begin{block}{Commonly used ion species}
%% 				\begin{itemize}
%% 					\item $^8$Be$^{+}$
%% 					\item $^{24}$Mg$^{+}$, $^{25}$Mg$^{+}$
%% 					\item $^{40}$Ca$^{+}$, $^{43}$Ca$^{+}$
%% 					\item $^{88}$Sr$^{+}$
%% 					\item $^{111}$Cd$^{+}$
%% 					\item $^{137}$Ba$^{+}$
%% 					\item $^{171}$Yb$^{+}$
%% 					\item $^{198}$Hg$^{+}$	       
%% 				\end{itemize}
%% 			\end{block}
%% 		\end{column}
%% 		\begin{column}{7.5cm}
%% 			\begin{center}
%% 				\includegraphics[width=7.5cm]{pictures/Periodic_Table_iontrap.eps}
%% 			\end{center}
%% 		\end{column}
%% 	\end{columns}
%% }

%% \frame
%% {
%% 	\frametitle{Calcium for ion trapping}
%% 	\begin{columns}[T]
%% 		\begin{column}{5cm}
%% 			\begin{block}{Calcium advantages}
%% 				\begin{itemize}[<+->]
%% 					\item All transitions accessible\\ with diode lasers.
%% 					\item Long-lived metastable\\ state.
%% 					\item Different isotopes available ($^{40}$Ca,$^{43}$Ca,...)
%% 				\end{itemize}
%% 			\end{block}
%% 		\end{column}
%% 		\begin{column}{5cm}
%% 			\includegraphics[height=4.5cm]{pictures/ca40ionlevels_v3.eps}%
%% 		\end{column}
%% 	\end{columns}

%% }


%% \section[Linear Paul trap]{Experiments with a macroscopic linear Paul trap}
%% \frame
%% {
%% 	\frametitle{Experiments with a Paul trap}
%% 	\begin{columns}[c]
%% 		\begin{column}{6cm}
%% 			\begin{itemize}
%% 				\item<2-> Deterministic entanglement\\ \begin{scriptsize}[Home, New J. Phys. 8, 188 (2006)]\end{scriptsize}
%% 				\item<3-> Mesoscopic Schr\"odinger's cat state\\ \begin{scriptsize}[McDonnell, PRL 98, 063603 (2007).]\end{scriptsize}
%% 				\item<4-> $>$1s coherence qubit\\ \begin{scriptsize}[Lucas, arXiv pre-print:0710.4421]\end{scriptsize}
%% 				\item<5-> High fidelity readout\\ \begin{scriptsize}[Myerson, PRL 100, 200502 (2008)]\end{scriptsize}
%% 				\item<6-> Sympathetic cooling\\ \begin{scriptsize}[Home, arXiv pre-print:0810.1036]\end{scriptsize}
%% 			\end{itemize}
%% 		\end{column}
%% 		\begin{column}{4cm}
%% 			\includegraphics<1->[height=4.5cm]{pictures/ourpaultrap.eps}%
%% 		\end{column}
%% 	\end{columns}
%% }

%% \frame
%% {
%% 	\frametitle{Towards large-scale quantum computing}
%% 	\begin{columns}[c]
%% 		\begin{column}{6cm}
%% 			\begin{block}{How to scale up?}
%% 				\begin{itemize}
%% 					\item<1-> Squeezing more ions into single trapping region
%% 					\item<2-> Using separate trapping regions
%% 				\end{itemize}
%% 			\end{block}
%% 		\end{column}
%% 		\begin{column}{5cm}
%% 			\includegraphics<1>[width=5cm]{pictures/manyions.eps}%
%% 			\includegraphics<2>[width=4cm]{pictures/traparray.eps}%
%% 		\end{column}
%% 	\end{columns}
%% }


%% \section[Sandia trap]{A microscopic design: the Sandia trap}
%% \frame
%% {
%% 	\frametitle{The Sandia design}
%% 	\begin{columns}[c]
%% 		\begin{column}{6cm}
%% 			\begin{block}{Design aims and issues}
%% 				\begin{itemize}
%% 					\item Aim: prototype multiple trapping region design
%% 					\item Semi-planar arangement of electrodes
%% 					\item Precision manufacturing with microfabrication
%% 					\item RF wiring needed manual modification
%% 					\item Added out-of-plane wire electrode
%% 				\end{itemize}
%% 			\end{block}
%% 		\end{column}
%% 		\begin{column}{5cm}
%% 			\includegraphics[width=5cm]{pictures/sandiatrapscheme2.eps}%
%% 		\end{column}
%% 	\end{columns}
%% }

%% \subsection{Design}
%% \frame
%% {
%% 	\frametitle{The Sandia design}
%% 	\begin{columns}[c]
%% 		\begin{column}{4.5cm}
%% 			\begin{block}{Electric field in the sandia trap}
%% 				\begin{itemize}
%% 					\item RF electrodes shield the ion from DC
%% 					\item Needs out-of-plane elements for complete control:\\ ground plane, extra wire-electrode
%% 					\item Good uniformity of RF field along the axis
%% 				\end{itemize}
%% 			\end{block}
%% 		\end{column}
%% 		\begin{column}{6cm}
%% 			\begin{center}
%% 				\includegraphics[width=6cm]{pictures/sandiafield_v1.eps}\\ Radial cross section of the RF field at the trap centre
%% 			\end{center}
%% 		\end{column}
%% 	\end{columns}
%% }

%% \subsection{Micromotion compensation}
%% \frame
%% {
%% 	\frametitle{Micromotion compensation}
%% 	\begin{columns}[c]
%% 		\begin{column}{5cm}
%% 			\begin{block}{}
%% 				\begin{itemize}
%% 					\item<1-> Need to overlap RF saddle point and DC minimum
%% 					\item<1-> Adjustments to voltages on DC electrodes
%% 					\item<1-> "Bad" compensation:\\ \alert{scan over atomic transition}
%% 					\item<2-> The finishing touches:\\ \alert{RF-photon correlation}
%% 				\end{itemize}
%% 			\end{block}
%% 		\end{column}
%% 		\begin{column}{6cm}
%% 			\includegraphics[width=6cm]{pictures/compensation397_2.eps}%
%% 		\end{column}
%% 	\end{columns}
%% }

%% \frame
%% {
%% 	\frametitle{RF-photon correlation}
%% 	\begin{columns}[c]
%% 		\begin{column}{5cm}
%% 			\begin{block}{Sequence}
%% 				\begin{itemize}
%% 					\item Function generator provides RF + sync signal
%% 					\item Photon counting with photomultiplier (PMT)
%% 					\item Time-to-amplitude converter (TAC) for timing of photons within RF cycle
%% 					\item Computer generate hystograms from TAC voltage
%% 				\end{itemize}
%% 			\end{block}
%% 		\end{column}
%% 		\begin{column}{6cm}
%% 			\includegraphics[width=6cm]{pictures/computer_2}%
%% 		\end{column}
%% 	\end{columns}
%% }


%% \frame
%% {
%% 	\frametitle{RF-photon correlation}
%% 	\includegraphics[width=10.5cm]{rfcorr/rfexplots.eps}%
%% }

%% \frame
%% {
%% 	\frametitle{Heating rate measurement}
%% 	\begin{block}{Heating rate }
%% 		\begin{itemize}
%% 			\item Heating can cause decoherence of the quantum bit
%% 			\item \alert{Source} of heating is the \alert{changing electric field}
%% 			\item Noise on voltage line can be filtered
%% 			\item Stray potentials on the electrodes are hard to control
%% 			\item Heating rate is characteristic to the design
%% 		\end{itemize}
%% 	\end{block}
%% }

%% \subsection{Heating rate measurements}
%% \frame
%% {
%% 	\frametitle{Doppler recooling method}
%% 	\begin{columns}[c]
%% 		\begin{column}{4.5cm}
%% 			\begin{block}{Experimental sequence}
%% 				\begin{itemize}
%% 					\item<1-> \alert<1>{Doppler-cooling} lasers are \alert<1>{on}.
%% 					\item<2-> \alert<2>{866nm} "repumping" laser turned \alert<2>{off}. Ion heats up.
%% 					\item<3-> \alert<3>{866nm} laser turned back \alert<3>{on}, time-resolved fluorescence recorded.
%% 				\end{itemize}
%% 			\end{block}
%% 		\end{column}
%% 		\begin{column}{5cm}
%% 			\includegraphics<1>[height=5.5cm]{heating/dopplerrecool_1.eps}%
%% 			\includegraphics<2>[height=5.5cm]{heating/dopplerrecool_2.eps}%
%% 			\includegraphics<3>[height=5.5cm]{heating/dopplerrecool_3.eps}%
%% 		\end{column}
%% 	\end{columns}
  
%% }

%% \frame
%% {
%% 	\frametitle{Experimental data}
%% 	\begin{columns}[c]
%% 		\begin{column}{5cm}
%% 			\begin{block}{Example scan}
%% 				\begin{itemize}
%% 				    \item Timing:\\ 50$\mu$s bin times
%% 					\item Fitted temperature:\\ 31.3$\pm$2.0 K 
%% 					\item Heating delay: 505ms
%% 					\item 301 repeats
%% 				\end{itemize}
%% 			\end{block}
%% 		\end{column}
%% 		\begin{column}{6cm}
%% 			\includegraphics[width=6cm]{heating/cooling_fit_v3.eps}
%% 		\end{column}
%% 	\end{columns}
%% }


%% \frame
%% {
%% 	\frametitle{Measurements}
%% 	\begin{columns}[c]
%% 		\begin{column}{5cm}
%% 			\begin{block}{Parameters affecting observed signal}
%% 				\begin{itemize}
%% 					\item<1-> Temperature:\\change in fluorescence
%% 					\item<2-> Laser detuning:\\change in shape
%% 					\item<3-> Laser intensity:\\change in time scale
%% 				\end{itemize}
%% 			\end{block}
%% 		\end{column}
%% 		\begin{column}{6cm}
%% 			\includegraphics<1>[width=6cm]{heating/coolingfit_theory2.eps}%
%% 			\includegraphics<2>[width=6cm]{heating/coolingfit_theory.eps}%
%% 			\includegraphics<3>[width=6cm]{heating/coolingfit_theory3.eps}%
%% 		\end{column}
%% 	\end{columns}
%% }

%% \frame
%% {
%% 	\frametitle{Effect of filtering}
%% 	\begin{columns}[c]
%% 		\begin{column}{5cm}
%% 			\begin{block}{Comparing heating rates}
%% 				\begin{itemize}
%% 					\item No filter on DC lines:\\ 206$\pm$80 K/s
%% 					\item Low-pass filter\\(cutoff $\sim$1kHz):\\ 50-60 K/s
%% 					\item Maximum heating delay of 705ms with filtering
%% 				\end{itemize}
%% 			\end{block}
%% 		\end{column}
%% 		\begin{column}{6cm}
%% 			\includegraphics[width=6cm]{heating/filterheating_v5.eps}%
%% 		\end{column}
%% 	\end{columns}
%% }

%% \frame
%% {
%% 	\frametitle{Compare to other traps}
%% 	\begin{block}{Noise spectral density vs. trap size}
%% 		\includegraphics[width=11.5cm]{pictures/noise_spectra_v3.eps}%
%% 	\end{block}
%% }


%% \subsection{Ion shuttle}

%% \frame
%% {
%% 	\frametitle{Ion shuttle}
%% 	\begin{columns}[c]
%% 		\begin{column}{5.5cm}
%% 			\begin{block}{Control voltages}
%% 				\begin{itemize}
%% 					\item Calculate electric field in the trap based on design (+corrections)
%% 					\item Predict voltage sequence to move ion to arbitrary position in trap
%% 					\item Estimate the effect of imperfect voltages
%% 				\end{itemize}
%% 			\end{block}
%% 		\end{column}
%% 		\begin{column}{5.5cm}
%% 			\includegraphics[width=5.1cm]{longshuttle/shuttlewaveform_v1.eps}\\
%% 			\includegraphics[width=4cm]{longshuttle/recordedwaveform_v1.eps}%
%% 		\end{column}
%% 	\end{columns}
%% }

%% \frame
%% {
%% 	\frametitle{Move that ion}
%% 	\begin{columns}[c]
%% 		\begin{column}{5.5cm}
%% 			\begin{block}{Example shuttle}
%% 				\begin{itemize}
%% 					\item Ion leaves the camera's view ($\pm50\mu m$ from trap centre)
%% 					\item Shuttles to 360$\mu m$ away
%% 					\item Delay time at far position: 120ms
%% 					\item Shuttle back to trap centre
%% 					\item Ion enter the camera's view, flurescence restored
%% 					\item Repetition rate is limited by software
%% 				\end{itemize}
%% 			\end{block}
%% 		\end{column}
%% 		\begin{column}{5.5cm}
%% 			\includegraphics[width=5.1cm]{longshuttle/longshuttle_v1.eps}\\
%% 		\end{column}
%% 	\end{columns}
%% }


%% \section[Other traps]{Other ion traps at Oxford}
%% \frame
%% {
%% 	\frametitle{Liverpool trap}
%% 	\begin{columns}[c]
%% 		\begin{column}{6cm}
%% 			\begin{block}{Summary}
%% 				\begin{itemize}
%% 					\item Mesoscopic distance scales\\ (750$\mu m$ Electrode-ion separation)
%% 					\item Manual assembly (with problems)
%% 					\item Trapped ions
%% 					\item Heating rate ($\sim 8$K/s) large for this trap size
%% 					\item "Mysterious" compensation
%% 				\end{itemize}
%% 			\end{block}
%% 		\end{column}
%% 		\begin{column}{5cm}
%% 			\includegraphics<1>[width=5cm]{pictures/liverpool_trap.eps}%
%% 			\includegraphics<2>[width=5cm]{pictures/liverpool3.eps}%
%% 		\end{column}
%% 	\end{columns}
%% }


%% \frame
%% {
%% 	\frametitle{The Liverpool design}
%% 	\begin{columns}[c]
%% 		\begin{column}{6.5cm}
%% 			\begin{center}
%% 				\includegraphics[width=6.5cm]{pictures/liverpoolsheme_v3.eps}\\
%% 				Schematics
%% 			\end{center}
%% 		\end{column}
%% 		\begin{column}{4.5cm}
%% 			\begin{center}
%% 				\includegraphics[width=4cm]{pictures/liverpool_contours_v1_4.eps}\\
%% 				Radial cross-section of RF field
%% 			\end{center}
%% 		\end{column}
%% 	\end{columns}
%% }

%% \frame
%% {
%% 	\frametitle{The Lucent design}
%% 	\begin{columns}[c]
%% 		\begin{column}{6cm}
%% 			\begin{block}{Lucent trap results}
%% 				\begin{itemize}
%% 					\item<1-> Large number of trapping regions
%% 					\item<1-> Small/variable length scale 
%% 					\item<1-> Microfabrication
%% 					\item<2-> \alert{Except}: it didn't work...
%% 				\end{itemize}
%% 			\end{block}
%% 		\end{column}
%% 		\begin{column}{5cm}
%% 			\begin{center}
%% 				\includegraphics[width=5cm]{pictures/LucentTrapSmall.eps}\\
%% 			\end{center}
%% 		\end{column}
%% 	\end{columns}
%% }


%% \section[Outlook]{Outlook}
%% \frame
%% {
%% 	\frametitle{Outlook}
%% 	\begin{block}{Many different directions}
%% 		\begin{itemize}
%% 			\item Experiments with more than two ions in Paul trap (quantum gates)
%% 			\item Probing fundamental physics:\\ "weak measurement"
%% 			\item Separating/joining ions
%% 			\item Different trap designs and trap manufacturing:\\ surface trap, ...
%% 		\end{itemize}
%% 	\end{block}
%% }

%% \frame
%% {
%% 	\frametitle{The surface trap}
%% 	\begin{block}{Manufacturing "theory and practice"}
%% 		\begin{columns}[T]
%% 			\begin{column}{5.5cm}
%% 				\includegraphics<1>[height=4cm]{pictures/surface_layout.eps}
%% 				\includegraphics<2>[width=4cm]{pictures/bad_surface.eps}
%% 			\end{column}
%% 			\begin{column}{5.5cm}
%% 				\includegraphics<1->[width=4cm]{pictures/surface_mic_1.eps}
%% 			\end{column}
%% 		\end{columns}
%% 	\end{block}
%% }


%% \frame
%% {
%% 	\frametitle{The Oxford Ion Trapping Group}
%% 	\includegraphics[width=10.5cm]{pictures/groupphoto.eps}%
%% }


\end{document}
