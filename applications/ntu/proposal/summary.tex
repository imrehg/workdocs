% summary.tex

\section{Summary}

Rydberg atomic systems emerged in the last decade as powerfool tools for quantum computing and simulation. Rydberg ions, however, could further open the field by using estabished experimental techniques available in ion trapping, and leveraging interactions that are not found in the atomic systems. There are theoretical proposals to create Rydberg ions and to implement quantum gates to manipulate them, but thus far no experimental realization was achieved. This proposal argues calls for building up an ion trap system and creation of Rydberg ions. These systems have the potential on the medium and long term to test Rydberg physics in external fields, to extend the standard ion trap toolbox, and provide model system to simulate a large variety of interacting Hamiltonians.


% \subsection*{Intellectual metrits}

% \subsection*{Broader implications}
