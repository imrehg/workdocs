% summary.tex

\section{Summary}

Rydberg atomic systems emerged in the last decade as powerful tools for quantum computing and simulation. Rydberg ions could further open the field by leveraging established experimental techniques available in ion trapping, and interactions that are not found in trapped neutral atoms. Theoretical proposals exists for the creation of such ions and for one- and two-ion manipulation, but so far they have not been experimentally realized. This proposal argues for building up an ion trap system that is designed to accommodate the theoretically outlined methods and experimentally study Rydberg ions. These systems have the potential to test Rydberg physics in external fields, extend the standard ion trap toolbox, implement fast quantum gates and provide model system to simulate a large variety of interacting Hamiltonians.
