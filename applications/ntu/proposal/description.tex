\section{Project description}

\subsection{Research objectives}

We want to build an ion trap that is capable of creating holding and manipulating Rydberg ions. Rydberg atoms are common and active field of study. 

Review paper: \cite{Saffman2010}, proposal: \cite{Mueller2008}.

\subsubsection{Intellectual merit of the work}

Currently there exist only theoretical proposal on how to create and manipulate Rydberg ions. While many potential advantages are outlined, there are a number of questions as well. According to the theoretical proposal, once the technical details are smoothed out, there are a large number of different studies that one can done.

\subsubsection{Broader impact of proposed work}

Rydberg ions have the potential to become practical and versatile tools in general quantum simulation problems. The large number of available parameters can describe many different physical systems, and the researchers have large freedom to manipulate those parameters. One would be able to test varios interaction Hamiltonians. Also, a different way of quantum information processing: collective, fast.

\subsection{Technical section}

\subsection{Schedule}

\subsection{Long term outlook}

Testing different ion species (different isotopes). Mixed systems (sympathetic cooling, different interaction hamiltonian). Different trap geonetry (ring trap instead of linear). Mixed ion and rydberg quantum computing for large scale systems: ion internal storage for transport, rydberg for manipulation.
